%% -*- coding: utf-8 -*-
\documentclass[12pt,a4paper]{scrartcl} 
\usepackage[utf8]{inputenc}
\usepackage[english,russian]{babel}
\usepackage{indentfirst}
\usepackage{misccorr}
\usepackage{graphicx}
\usepackage{amsmath}
\usepackage{float}
\usepackage{ dsfont }

\usepackage{xcolor}
\usepackage{hyperref}
\hypersetup{colorlinks,
  pdftitle={The title of your document},
  pdfauthor={Your name},
  allcolors=[RGB]{000 000 000}}

\begin{document}
\begin{titlepage}
  \begin{center}

    Санкт-Петербургский политехнический университет Петра Великого

    \vspace{0.25cm}
    
    Институт прикладной математики и механики
    
    Кафедра «Прикладная математика»
    \vfill

	\vspace{0.25cm}
	    Отчёт\\
	по лабораторной работе №4\\
	по дисциплине\\
	«Вычислительные комплексы»

  \bigskip

\end{center}
\vfill

\newlength{\ML}
\settowidth{\ML}{«\underline{\hspace{0.7cm}}» \underline{\hspace{2cm}}}
\hfill\begin{minipage}{0.4\textwidth}
  Выполнил студент\\ В.\,А.~Рыженко\\
\end{minipage}%
\bigskip

\hfill\begin{minipage}{0.4\textwidth}
  Проверил:\\
к.ф.-м.н., доцент\\
Баженов Александр Николаевич\\
\end{minipage}%
\vfill

\begin{center}
  Санкт-Петербург, 2020 г.
\end{center}
\end{titlepage}

\tableofcontents
%\listoffigures
\newpage


\section{Постановка задачи}

Требуется решить ИСЛАУ с применением аппарата линейного программирования для проведения регуляризации рассматриваемой системы. 


\section{Конкретизация задачи и метод решения}
При решении данной задачи имеет рассмотреть ИСЛАУ $Ax=b$ точечной марицей $A$ и
интервальной правой частью $\textbf{b}$ при которых система не имеет решений до проведения регуляризации. В данной работе выбрана несовместная ИСЛАУ:
\begin{equation}
    \begin{pmatrix}
    2 && 3 && 4 \\
    2.2 && 3 && 4 \\
    9 && 0 && 0 \\
    \end{pmatrix} \cdot x = 
    \begin{pmatrix}
    [2; 4] \\
    [7; 9] \\
    [-1; 1] \\
    \end{pmatrix}
    \label{islau}
\end{equation}
В первую очередь с помощью распознающего функционала $Tol(x)$ проверяется отсутствие решений у данной системы. С помощью программы
tolsolvty были найдены максимум функционала распознающего функционала $maxTol$ и
значение аргумента, в которой он достигался $argmaxTol$:
\begin{equation}
    maxTol = -1.4725;
    argmaxTol = \begin{pmatrix}
    0.2747 \\
    0.5907 \\
    0.7876
    \end{pmatrix}
    \label{results}
\end{equation}
Поскольку $maxTol < 0$, допусковое множество ИСЛАУ пусто и система несовместна. \\
Далее для получения решения проводится $l_{1}$-регуляризация, заключающуюся в изменении радиусов компонент вектора $\textbf{b}$ их поэлементным домножением на вектор масштабирующих множителей $\omega$:
\begin{equation}
    \textbf{b} = \begin{pmatrix}
    [\text{mid} b_{1} - \text{rad} b_{1}; \text{mid} b_{1} + \text{rad} b_{1}] \\
    [\text{mid} b_{2} - \text{rad} b_{2}; \text{mid}b_{2} + \text{rad} b_{2}] \\
    [\text{mid} b_{3} - \text{rad} b_{3}; \text{mid} b_{3} + \text{rad} b_{3}] \\
    \end{pmatrix} \rightarrow \tilde{\textbf{b}} = \begin{pmatrix}
    [\text{mid} b_{1} - \omega_{1}\text{rad} b_{1}; \text{mid} b_{1} + \omega_{1}\text{rad} b_{1}] \\
    [\text{mid} b_{2} - \omega_{2}\text{rad} b_{2}; \text{mid} b_{2} + \omega_{2}\text{rad} b_{2}] \\
    [\text{mid} b_{3} - \omega_{3}\text{rad} b_{3}; \text{mid} b_{3} + \omega_{3}\text{rad} b_{3}] \\
    \end{pmatrix}
    \label{after_l1_reg}
\end{equation}
При этом масштабирующие множители подбираются так, чтобы регуляризованная ИСЛАУ $A \cdot x = \tilde{\textbf{b}}$ стала разрешима, но сумма этих множителей $\sum_{i}{\omega_{i}}$ была минимально возможной. \\
Накладывая на масштабирующие множители естественное требование их неотрицательности, и введя вектор $u = \begin{pmatrix}x \\ \omega\end{pmatrix}$, можно записать полученную задачу в виде:
\begin{equation}
    \begin{cases}
    u_{4, 5, 6} \geq 0 \\
    c \cdot u = (0,0,0,1,1,1) \cdot u= (0,0,0,1,1,1) \cdot \begin{pmatrix}
    x_{1} \\ x_{2} \\ x_{3} \\ \omega_{1} \\ \omega_{2} \\ \omega_{3}
    \end{pmatrix} = \sum_{i}{\omega_{i}} = \underset{u}{min} \\
    C \cdot u \leq r, \text{где } C = \begin{pmatrix}
    -A & -\text{diag}(\text{rad}(\textbf{b})) \\
    A & -\text{diag}(\text{rad}(\textbf{b}))
    \end{pmatrix}, r = \begin{pmatrix}
    -mid(\textbf{b}) \\
    mid(\textbf{b})
    \end{pmatrix}
    \end{cases}
    \label{system}
\end{equation}
Полученная задача и решается линейным программированием с применением стандартной
функции linprog пакета scipy.optimize. В результате решения определяются одновременно и
необходимые масштабирующие множители, и соответствующее им появившееся в результате
регуляризации решения ИСЛАУ.


\section {Реализация}
Лабораторная работа выполнена с помощью встроенных средств языка программирования Matlab и Python. Также используется оптимизатор scipy.optimize на Python с различными методами решения задачи линейного программирования.
Для нахождения экстремума распознающего функционала использована программа tolsolvty для Python.

\section {Результаты}

Имеем rad$(\textbf{b}) = 1, $mid$(\textbf{b}) = \begin{pmatrix}3 & 8 & 0\end{pmatrix}$.\\
По:
\begin{equation}
    C = \begin{pmatrix}
    -2 & -3 & -4 & -1 & 0 & 0 \\
    -2.2 & -3 & -4 & 0 & -1 & 0 \\
    -9 & 0 & 0 & 0 & 0 & -1 \\
    2 & 3 & 4 & -1 & 0 & 0 \\
    2.2 & 3 & 4 & 0 & -1 & 0 \\
    9 & 0 & 0 & 0 & 0 & -1 \\
    \end{pmatrix};
    r = \begin{pmatrix}-3 & -8 & 0 & 3 & 8 & 0\end{pmatrix}
    \label{after_regul}
\end{equation}
В результате применения стандартного linprog для решения задачи линейного программирования с использованием значений из (\ref{after_regul}) без дополнительных ограничений получны следующие результаты:
\begin{itemize}
    \item Решение регуляризованной ИСЛАУ методом method = ’interior-point’:
    \begin{equation}
        x \approx \begin{pmatrix}
        0 \\ 0.6387 \\ 0.8661
        \end{pmatrix},
        \omega \approx \begin{pmatrix}
        2.3806 \\2.6194 \\ 0
        \end{pmatrix}
        \label{res_ip}
    \end{equation}
     \item Решение регуляризованной ИСЛАУ методом method = ’simplex’:
    \begin{equation}
        x \approx \begin{pmatrix}
        0 \\ 0 \\ 2.0
        \end{pmatrix},
        \omega \approx \begin{pmatrix}
        5 \\ 0 \\ 0 
        \end{pmatrix}
        \label{res_simpl}
    \end{equation}
\end{itemize}

Можно отметить, что сумма масштабирующих коэффициентов совпала и равна 5. \\
Также, первые компоненты вектора решений оказались равны 0, что  наводит на мысль, что рассматриваемая задача может иметь нетривиальное множество одинаково хороших решений. Попробуем наложить дополнительные ограничения. 
\begin{equation}
x_{i min} \leq x_i \leq x_{i max}
\end{equation}

В результате сначала было обнаружено, что если установить нижнюю границу возможных
значений $x_1$, большую, чем ноль , $x_{1min}$> 0, то условная оптимизация сходится к $x_1 = x_{1min}$, а значение суммы масштабирующих коэффициентов увеличивается и становится больше 5. Это говорит о том, что решение c $x_1 = 0$ действительно является оптимальным и устойчивым.

%\begin{figure}[H]
%    \centering
%    \includegraphics{points_changing.png}
%    \caption{Изменения нижней границы одной координат решения в обоих методах}
%    \label{fig:change2}
%\end{figure}
%Стоит также отметить, что в случае изменения границ только в симплекс-методе, в определенный момент решения совпадают, причем по всем компонентам(в точке $(1.4 0.5807 0.6344)^{T}$).
%\begin{figure}[H]
%    \centering
%    \includegraphics[width=10cm, height=8cm]{2-nd simplex.png}
%    \caption{Изменение нижней границы второй координаты вектора решений для симплекс-метода}
%    \label{fig:simplex_change}
%\end{figure}
%В случае, если вносить изменения в границы 2-х компонент решения, будет заметно, что с той же точки решения будут совпадать:
%\begin{figure}[H]
%    \centering
%    \includegraphics{points_b_changing.png}
%    \caption{Изменение нижней границы обеих компонент вектора решений в обоих методах}
%    \label{fig:all_change}
%\end{figure}
%Таким образом, можно заключить, что существует множество  решений поставленной задачи линейного программирования, соответствующее фиксированному значению $x_{1} = 1.4$ и
%целой полосе возможных значений по остальным компонентам.


\section{Приложения}
Репозиторий на GitHub с релизацией: \href{https://github.com/WiillyWonka/Intervals}{github.com}.
\end{document}

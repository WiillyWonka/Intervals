%% -*- coding: utf-8 -*-
\documentclass[12pt,a4paper]{scrartcl} 
\usepackage[utf8]{inputenc}
\usepackage[english,russian]{babel}
\usepackage{indentfirst}
\usepackage{misccorr}
\usepackage{graphicx}
\usepackage{amsmath}
\usepackage{float}
\usepackage{ dsfont }

\usepackage{xcolor}
\usepackage{hyperref}
\hypersetup{colorlinks,
  pdftitle={The title of your document},
  pdfauthor={Your name},
  allcolors=[RGB]{000 000 000}}

\begin{document}
\begin{titlepage}
  \begin{center}

    Санкт-Петербургский политехнический университет Петра Великого

    \vspace{0.25cm}
    
    Институт прикладной математики и механики
    
    Кафедра «Прикладная математика»
    \vfill

	\vspace{0.25cm}
	    Отчёт\\
	по лабораторной работе №1\\
	по дисциплине\\
	«Вычислительные комплексы»

  \bigskip

\end{center}
\vfill

\newlength{\ML}
\settowidth{\ML}{«\underline{\hspace{0.7cm}}» \underline{\hspace{2cm}}}
\hfill\begin{minipage}{0.4\textwidth}
  Выполнил студент\\ В.\,А.~Рыженко\\
\end{minipage}%
\bigskip

\hfill\begin{minipage}{0.4\textwidth}
  Проверил:\\
к.ф.-м.н., доцент\\
Баженов Александр Николаевич\\
\end{minipage}%
\vfill

\begin{center}
  Санкт-Петербург, 2020 г.
\end{center}
\end{titlepage}

\tableofcontents
%\listoffigures
\newpage


\section{Постановка задачи}

\subsection{Задача 1}

Имеем 2х2 матрицу A~\eqref{eq:A_base}

\begin{equation}\label{eq:A_base}
\centering
\begin{pmatrix}
1 & 1\\
1.1 & 1
\end{pmatrix}
\end{equation}

Пусть все элементы матрицы $a_{ij}$ имеют теперь радиус $\varepsilon$:

\begin{equation}
\centering
rad \textbf{a}_{ij} = \varepsilon.
\end{equation}

Получаем 

\begin{equation}\label{eq:A_mod}
\centering
\begin{pmatrix}
[1 - \varepsilon, 1 + \varepsilon] & [1 - \varepsilon, 1 + \varepsilon]\\
[1.1 - \varepsilon, 1.1 + \varepsilon] & [1 - \varepsilon, 1 + \varepsilon]
\end{pmatrix}
\end{equation}

Определить, при каком радиусе $\varepsilon$ матрица ~\eqref{eq:A_mod} содержит особенные матрицы.

\subsection{Задача 2}

Имеем nхn матрицу A~\eqref{eq:A1}
$\begin{pmatrix}\label{eq:A1}
1 & [0, \varepsilon] & ... & [0, \varepsilon] \\
[0, \varepsilon] & 1 & ... & [0, \varepsilon] \\
&...&... \\
[0, \varepsilon] & [0, \varepsilon]& ... & 1
\end{pmatrix}
$
Определить, при каком радиусе $\varepsilon$ матрица ~\eqref{eq:A1} содержит особенные матрицы.

\section {Теория}

\subsection {Определение}

Интервальная матрица $\textbf{A} \in \mathds{I} \mathds{R}^{nxn} $называется неособенной, если
неособенны все точечные матрицы$ A \in \textbf{A}$. Интервальная матрица называется особенной, если она содержит
особенную точечную матрицу.

\subsection {Теорема}\label{Theorem}

Теорема. Пусть интервальная матрица $\textbf{A} \in \mathds{I} \mathds{R}^{nxn}$ такова, что её
середина $mid \textbf{A}$ неособенна и 

\begin{equation}
\centering
\max_{1\leq j \leq n}(rad  \textbf{A} \cdot |(mid( \textbf{A})^{-1}|)_{jj} \geq 1
\end{equation}

Тогда \textbf{A} — особенная.

\subsection {Теорема. (признак Бекка).}\label{Bek}

Пусть интервальная матрица$ \textbf{A} \in \mathds{I} \mathds{R}^{nxn} $такова, что ее середина $mid \textbf{A}$ неособенна и 

\begin{equation}
\centering
\rho (rad  \textbf{A} \cdot (mid( \textbf{A})^{-1}) < 1
\end{equation}

Тогда \textbf{A} неособенна.

\section {Реализация}
Лабораторная работа выполнена с помощью встроенных средств языка программирования Python в среде разработки Visual Code. Исходный код лабораторной
работы приведён в приложении.
 
\section{Результаты}

\subsection{Задача 1}

Для решения задачи воспользуемся \ref{Theorem}. Для $\textbf{A}$ имеем

\begin{equation*}
\centering
mid \textbf{A} = 
\begin{pmatrix}
1 & 1\\
1.1 & 1
\end{pmatrix}
\end{equation*}

\begin{equation*}
\centering
rad \textbf{A} = \begin{pmatrix}
\varepsilon & \varepsilon\\
\varepsilon & \varepsilon
\end{pmatrix}
\end{equation*}

$det (mid \textbf{A}) \neq 0 $, следовательно матрица удовлетворяет условию теоремы.

Получаем следующее:
	
\begin{equation*}
\centering
rad  \textbf{A} \cdot (mid( \textbf{A})^{-1} =
 \begin{pmatrix}
\varepsilon & \varepsilon\\
\varepsilon & \varepsilon
\end{pmatrix}
 \cdot
 \begin{pmatrix}
10 & 10\\
11 & 10
\end{pmatrix}
 = 
 \begin{pmatrix}
21 \varepsilon & 20 \varepsilon\\
21 \varepsilon & 20 \varepsilon
\end{pmatrix}
\end{equation*}

Отсюда получаем, что матрица будет особенной при $\varepsilon \geq \frac{1}{21}$
\newline

Проверим с помощью программы определитель, получим следующее:
\begin{flushleft}
  Enter  eps:\\0.048\\det = (-0.2968, 0.0968)
\end{flushleft}
Уточним нижнюю границу, сдвигая $\varepsilon$ на $\Delta \varepsilon = -0.001$ю
Получим слудующий результат:
\begin{flushleft}
end eps = 0.025\\det = (-0.2025, 0.0025)
\end{flushleft}

\subsection{Задача 2}

Рассмотри решение на примере матрицы 3х3. Применим криетрий Бека\ref{Bek} и получим следующий результат:
\begin{flushleft}
end eps = 1.29\\pho = 4.657\\det = (-0.6641, 1)
\end{flushleft}
Уточним полученное значение аналогично прошлой задаче, получим:
\begin{flushleft}
end eps = 0.58\\det = (-0.0092, 1.3902)
\end{flushleft}

\section{Приложения}
Репозиторий на GitHub с релизацией: \href{https://github.com/WiillyWonka/Intervals}{github.com}.
\end{document}
